\documentclass[12pt]{article}

\usepackage{times}            % PostScript Times-Roman Font
\usepackage{epsfig}

\setlength{\paperheight}{297mm}
\setlength{\paperwidth}{210mm}
\setlength{\textwidth}{170mm}
\setlength{\textheight}{245mm}
\setlength{\voffset}{0mm}
\setlength{\topmargin}{9.6mm}         % 25.4mm + 9.6mm = 35mm
\setlength{\headheight}{0mm}
\setlength{\headsep}{0mm}
\setlength{\footskip}{0mm}
\setlength{\hoffset}{0mm}
\setlength{\oddsidemargin}{-5.4mm}    % 25.4mm - 5.4mm = 20mm
\setlength{\evensidemargin}{-5.4mm}   % 25.4mm - 5.4mm = 20mm
\setlength{\parindent}{0mm}  
\setlength{\parskip}{1ex}  

\pagestyle{empty}
\renewcommand{\baselinestretch}{0.9}
\newcommand{\RULE}{\rule[-1mm]{0mm}{5mm}}

\begin{document}
\vspace*{-25mm}
\centerline{\small Journal of Biomechanics Abstract 2016}
\vspace{1mm}\hrule\vspace{12pt}
%
\vspace{10mm}
{\centering \large \bf
KINEMATICALLY SIMILAR BASKETBALL FREE THROWS HAVE SURPRISINGLY DIFFERENT MUSCLE CONTRACTION VELOCITY PROFILES \par}

\vspace{3mm}
\normalsize \rm
\centerline{Daniel A Hagen$^1$, *Steven Caja$^1$, *Suraj Chakravarthi$^2$, Francisco J Valero-Cuevas$^{1,3}$}
\normalsize 

\vspace{3mm}

{
\centering
$^1$University of Southern California, Department of Biomedical Engineering, Los Angeles, CA, USA \break
$^2$University of Southern California, Department of Electrical Engineering, Los Angeles, CA, USA \break
$^3$University of Southern California, Department of Biokinesiology and Physical Therapy, Los Angeles, CA, USA \break
E-mail: dhagen@usc.edu Website: valerolab.org \break
*Equal Contributions\break\par
}

\vspace{3mm}

\textbf{KEYWORDS}: biomechanics, motor control, basketball, free throw, fiber velocity

\vspace{3mm}

An accurate basketball shot is difficult to replicate. Often times, simple mimicry does not suffice to accomplish a professional level of accuracy. Recent work re-emphasizes that the neural control of limb movements is overdetermined with the rotations of a few joints determining the length changes in all muscles [1,2]. As pointed out by Sherrington [3], movement can be disrupted if even one muscle undergoing eccentric contraction fails to silence its stretch reflex appropriately. Thus throws requiring higher eccentric velocities require more accurate alpha-gamma control, and are thus more prone to variability. Higher concentric velocities also reduce muscle power output. Using a planar, 18 muscles, 3 $\textit{dof}$ model and 300,000 randomly generated joint angle trajectories for a 550 ms free throw duration, we explored the differences in muscle fiber velocity profiles created through posture specific moment arm values [1,4,5]. By comparing the sum of maximum contraction velocities squared for each muscle for both eccentrically and concentrically contracting muscles we observed that kinematically similar throws in the endpoint space can exhibit large differences in muscle contraction velocity profiles. 
\vspace{3mm}
\begin{flushleft}


\textbf{REFERENCES}\break
1. Valero-Cuevas, FJ, $\textit{Fundamentals of Neuromechanics}$, Springer-Verlag London, 2016.\break
2. Valero-Cuevas, F, Cohn, B, Yngvason, H, $\&$ Lawrence, E, $\textit{J Biomech}$ $\textbf{48}$(11), 2887-2896, 2015.\break
3. Sherrington, C.S. $\textit{Exp. Physiol.}$, $\textbf{6}$(3) 252-310, 1913.\break
4. Ramsay, JW, Hunter, BV, $\&$ Gonzalez, RV, $\textit{J Biomech}$, $\textbf{42}$(4), 463-473, 2009.\break
5. Holzbaur, KR, Murray, WM, $\&$ Delp, SL, $\textit{Annals of Biomedical Engineering}$, $\textbf{33}$(6), 829-840, 2005\break
\end{flushleft}


\textbf{ACKNOWLEDGEMENTS}\break
We thank the University of Southern California for facilities provided during the course BME/BKN 504 and Emily Lawrence who laid the foundation for this research and shared her prior work.

\end{document}